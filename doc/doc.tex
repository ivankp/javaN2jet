\documentclass[12pt]{article}
\usepackage[cm]{fullpage}
\usepackage{fancyhdr}
\usepackage{indentfirst}
\usepackage{upgreek}
\usepackage{enumitem}
\usepackage{amsmath}
\usepackage{graphicx}
\usepackage{float}
\usepackage{hyperref}
\hypersetup{colorlinks=true, linkcolor=black, citecolor=blue, urlcolor=blue}

\newcommand{\eq}[1]{\hyperref[eq:#1]{Eq.~(\ref{eq:#1})}}
\newcommand{\fig}[1]{\hyperref[fig:#1]{Fig.~\ref{fig:#1}}}
\newcommand{\tab}[1]{\hyperref[table:#1]{Table~\ref{table:#1}}}
\newcommand{\sect}[1]{\hyperref[sec:#1]{Section~\ref{sec:#1}}}
\newcommand{\app}[1]{\hyperref[sec:#1]{Appendix~\ref{sec:#1}}}

\newcommand{\red}[1]{\textcolor{red}{#1}}

\title{Java implementation of $O(N^2)$ jet clustering algorithms}
\author{Ivan Pogrebnyak, MSU, ANL}
\date{\today}

\makeatletter
  \pagestyle{fancy}
  \lhead{\@title}
  \rhead{\@date}
  \cfoot{\thepage}
  \rfoot{\@author}
\makeatother

\def\headheight{15pt}
\def\headsep{15pt}
\def\textheight{650pt}

\renewcommand{\footrulewidth}{0.4pt}

\def\abovecaptionskip{3pt}
\def\belowcaptionskip{2pt}

\def\textsim{\raise.25ex\hbox{$\scriptstyle\sim$}}

\begin{document}
%\maketitle

%%%%%%%%%%%%%%%%%%%%%%%%%%%%%%%%%%%%%%%%%%%%%%%%%%%%%%%%%%%%%%%%%%%%%

\section{Introduction}
The motivation for developing an optimized Java based jet clustering algorithm, is to provide a platform independent implementation for use in online validation scripts on HepSim\cite{HepSim} repository with Monte Carlo predictions for HEP experiments.

Currently, the most widely used implementation of jet clustering algorithms in high energy physics is FastJet\cite{fastjet_man}. Is it written in C++ and provides a highly optimized implementation of (anti)kt and Cambridge-Aachen jet finding algorithms with selectable $O(N^2)$ and $O(N\log N)$ complexities, which significantly reduce runtime in comparison to naive $O(N^3)$ algorithms\cite{fastjet_n3}.

Unfortunately, platform dependence of C++ is prohibitive for use of FastJet in client-side web application  scripts. Therefore, a Java implementation was necessary.

\section{Implementation}

It turns out, that for reconstruction of jets obtained in pp collisions, with at most a few thousand particles per event, the most efficient algorithms are those with $O(N^2)$ complexity.
The $O(N\log N)$ algorithms provide an advantage only in events with more then 10,000 particles, which only arise in experiments with heavy ion collisions\cite{fastjet_n3,fastjet_unpub}. While implementation of $O(N\log N)$ algorithms requires use of sophisticated techniques, like Voronoi diagrams\cite{Aurenhammer}, $O(N^2)$ complexity can be achieved with relatively simple approaches described below.

\subsection{Generalizaed kt algorithm}

\subsection{Geometric Factorization}
This is the most significant optimization, reducing complexity from $O(N^2)$ to $O(N^3)$.
It is explained by the FastJet Lemma\cite{fastjet_n3}.

{\vspace{7px} \bf FastJet Lemma 1:} If particles $i$, $j$ form the smallest $d_{ij}$, and $p_t$

{\vspace{7px} \bf Motivation:} 

{\vspace{7px} \bf Proof:} 

\subsection{Tiling}

{\vspace{7px} \bf FastJet Lemma 2:} 

{\vspace{7px} \bf Motivation:} 

{\vspace{7px} \bf Proof:} 

\subsection{Linked List}


\section{Benchmark}
\red{Make plots to compare to FastJet performarmance.}

\begin{thebibliography}{9}
  
\bibitem{HepSim}
  S.~V.~Chekanov,
  ``HepSim: a repository with predictions for high-energy physics experiments,''
  \href{http://arxiv.org/abs/1403.1886}{arXiv:1403.1886 [hep-ph]}.

\bibitem{fastjet_man}
  M.~Cacciari, G.~P.~Salam and G.~Soyez,
  ``FastJet user manual,''
  Eur.\ Phys.\ J.\ C {\bf 72} (2012) 1896
  [\href{http://arxiv.org/abs/1111.6097}{arXiv:1111.6097 [hep-ph]}].

\bibitem{fastjet_n3}
  M.~Cacciari and G.~P.~Salam,
  ``Dispelling the $N^{3}$ myth for the $k_t$ jet-finder,''
  Phys.\ Lett.\ B\ {\bf 641} (2006) 57
  [\href{http://arxiv.org/abs/hep-ph/0512210}{hep-ph/0512210}].

\bibitem{fastjet_unpub}
  G. P. Salam and M. Cacciari,
  ``Jet clustering in particle physics, via a dynamic nearest neighbour graph implemented with CGAL,''
  \href{http://www.lpthe.jussieu.fr/~salam/repository/docs/kt-cgta-v2.pdf}{LPTHE-06-02},
  unpublished note.
  
\bibitem{Aurenhammer}
  F. Aurenhammer,
  ``Voronoi diagrams: a survey of a fundamental data structure,''
  ACM Computing Surveys 23 (1991) 345.

\end{thebibliography}

\end{document}
